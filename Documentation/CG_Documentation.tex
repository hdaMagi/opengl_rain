\documentclass[a4paper,11pt]{report}

%-------##-------##-------##------- SHORTCUTS -------##-------##-------##-------%

%------- Korrigiert hier bitte eure Namen und sortiert euch nach Nachname korrekt ein
\newcommand{\ajp}{Jannis Priesnitz}
\newcommand{\amt}{Magarethe TBC}
\newcommand{\Autoren}{\amt, \ajp}


%----------------- PDF CONFIG ----------------- %
\pdfinfo{    
     /Title Computergraphics Dokumentation
     /Subject   OpenGL Rain    
     /Author  (\Autoren) 
%     /Keywords   (TBC)      
} 

\title{***WORKINGTITLE***Simulation von Regen}
\author{\Autoren}
\date{30.10.2016}



%----------------- PAKETE INKLUDIEREN ----------------- %

\usepackage{geometry} % Packet für Seitenrandabständex und Einstellung für Seitenränder
\usepackage[ngerman]{babel} % deutsche Silbentrennung

\usepackage{booktabs} %entzerrt die Tabellenzeilen und bietet verschieden dicke Unterteilungslinien
\usepackage{longtable} % Tabellen können sich nicht über mehrere Seiten 
\usepackage{graphicx} % kann LaTeX Grafiken einbinden



\usepackage[utf8]{inputenc} % Umlaute unter Mac werden automatisch gesetzt
\usepackage[T1]{fontenc} % Zeichenencoding
\usepackage{lmodern} % typographische Qualität 
\frenchspacing % Schaltet den zusätzlichen Zwischenraum ab
\usepackage{fix-cm}
\usepackage{hyperref} % verwandelt alle Kapitelüberschriften, Verweise aufs Literaturverzeichnis und andere Querverweise in PDF-Hyperlinks
\usepackage{color}
\usepackage{url}

%Ampere, Ohm, Volt etc...
\usepackage[binary-units=true]{siunitx}

\usepackage{bera}% optional: just to have a nice mono-spaced font for code
\usepackage{xcolor}

\usepackage{caption}%mehrere Bilder pro Figure
\usepackage{subcaption}
\usepackage{float} %damit die Bilder [H](Here) auf der richtige Stelle angezeigt werden

\usepackage{setspace}

%\usepackage[nottoc]{tocbibind}

% für Listings
\usepackage{listings}
\include{Misc/codedefinitions}
\newcommand{\inlcpp}[1]{\lstinline[style=customcpp]|#1|}
% todos
\usepackage{todonotes}
\usepackage{textcomp}
%Einheiten
\usepackage{units}


%----------------- FARBEN DEFINIEREN ----------------- %
\definecolor{gray}{gray}{0.95} % Listingsbackground

%----------------- LAYOUT SETZEN ----------------- %
\geometry{left=2cm, right=2cm, top=2.5cm, bottom=2cm}
\linespread {1.25}\selectfont %1.25 da er von Haus aus 1.2 ist und 1,25 * 1,2 = 1,5 isch

%-------##-------##-------##------- ANFANG INHALT -------##-------##-------##-------%
\begin{document}

\pagenumbering{roman} % Seitennummer

%----------------- DECKBLATT -----------------%
 %----------------- KONFIGURATION ----------------- %
\pagestyle{empty} % enthalten keinerlei Kopf oder Fuß 

%----------------- HDA FBI Logo ----------------- %
\begin{figure}[h]
	\centering
	\includegraphics[width=0.6\textwidth]{Abb/logo_fbi}
\end{figure}
%----------------- INHALT ----------------- %
\begin{center}
\Large Hochschule Darmstadt \\
\normalsize \textsc{- Fachbereich Informatik -} \\
\end{center}


\begin{center}
\normalsize
% Whitespace
\vspace{105 pt}

\setstretch{1.5 }
\Huge ***WORKINGTITLE***Simulation von Regenfvdv 
\normalsize
\vspace{65 pt}

Vorlesung Computergraphics\\
Dokumentation der Praktikumsaufgabe im Sommersemester 2016\\
Prof. Hergenröther

\vspace{75 pt}



\vspace{5 pt}
\Autoren \\
\vspace{115 pt}



\end{center}
 
%----------------- ABSTRACT -----------------%
 %----------------- KONFIGURATION ----------------- %
\pagestyle{empty} % enthalten keinerlei Kopf oder Fuß


\chapter*{Abtract} % (fold)
\label{cha:abtract}
	
Abstact
 
 
%----------------- VERZEICHNISSE -----------------%
\tableofcontents % Inhaltverzeichnis

% ----- Abbildungen ----- %
%\addcontentsline{toc}{section}{Abbildungsverzeichnis} % falls in Inhalsverzeichnis
\listoffigures

% ----- Tabellen----- %
% \addcontentsline{toc}{section}{Tabellenverzeichnis}  % falls in Inhalsverzeichnis
% \fancyhead[L]{Abbildungsverzeichnis / Abkürzungsverzeichnis} %Kopfzeile links
% \listoftables

% ----- Listings ----- %
% Listingverzeichnis soll im Inhaltsverzeichnis auftauchen
% \addcontentsline{toc}{section}{Listingverzeichnis}
% \fancyhead[L]{Abbildungs- / Tabellen- / Listingverzeichnis} %Kopfzeile links
\renewcommand{\lstlistlistingname}{Listingverzeichnis}
\lstlistoflistings

\pagestyle{plain} % zurueck setzen von roemische seitenanzahl

\pagenumbering{arabic}
%----------------- KAPITEL : EINFÜHRUNG  ----------------- %            % GLOBALE DOKUMENTATION
\include{Kapitel/Einfuehrung}                                           % 

%----------------- KAPITEL : Anhang  ----------------- %              	%
\chapter*{Anhang}
	\label{cha:appendum}
    \begin{itemize}
    \item noch nichts
    \end{itemize}

%-------##-------##-------##------- QUELLEN -------##-------##-------##-------%
\renewcommand{\bibname}{Externe Bibliotheken, technische Dokumente und Quellen}
\bibliography{quellen}
\bibliographystyle{alpha}

\end{document}\grid
